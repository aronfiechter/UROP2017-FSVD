\documentclass[11pt,a4paper,english]{article}
\usepackage[T1]{fontenc}
\usepackage[utf8]{inputenc}
\usepackage{babel}
\usepackage{geometry}
    \geometry{
    	left=2cm,
	right=2cm,
	includeheadfoot, top=1cm, bottom=1cm,
    headsep=2cm,
    footskip=2.5cm
    }
\usepackage{fancyhdr}
\usepackage{lipsum}
\usepackage{import}

% Extensions file
\import{"/Users/aron/Google Drive/TEMPLATES/"}{falcon.tex} % path filled by fasttex in .zprofile

% Setup for page layout (fancyhdr)
\fancyhf{}
\lhead{Aron Fiechter}
\chead{UROP 2017}
\rhead{\today}
\cfoot{\thepage}
\pagestyle{fancy}

\begin{document}

    {\centering\huge\textbf{Project report}\par}

    \vspace{1cm}

    \section{Clear description of generally what you want to do and why. (fsvd, etc.)}

    \section{Clear description of the method proposed and why. (envelopes, etc.)}
    
    \begin{quote}
    For the 2 previous, introductory concepts, feel free to get inspired by existing papers/guides/etc, just mention them.
    \end{quote}
    
    \section{A timeline of all steps and approaches done.}
    
    \subsection{You should mention all issues faced one by one.}
    \subsection{How you attempted to solve them initially and why it didn't work out.}
    \subsection{How you solved them, finally.}
    \subsection{For all issues not solved, mention them clearly.}
    \subsection{Why you didn't manage to solve them, what were the main difficulties.}
    \subsection{How would you have solved them if you had more item (drop ideas).}
    \subsection{Special cases you didn't have time to tackle and ideas for dealing with them.}
    \subsection{What would further work on this topic include.}
    
    \begin{quote}
	In general keep in mind that someone would ideally read your report and clearly understand what your problem was, how you dealt with it, which difficulties you faced, how did you tackle them (or not) and how your work could be extended.
    \end{quote}
    

\end{document}
